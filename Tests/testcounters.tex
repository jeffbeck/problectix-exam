\documentclass[bsz-mkk-exam,ka]{teacher}
\Titelo{KA}
\Titelu{Nr. 1}
\Datum{12.12.2016}
\Fach{Lernbereich Theorie}
\Klasse{M1KK3T}
\UserToken{BZ}
\gruppea
\begin{document}

%\BTone{final}
\BTtwo{draft}

\begin{examaufgabe}{\hlogpdia}{20}{KT}{PF}{HLOGP}

\xanhang{\hlogpdia}{r134a}{MOP-ap-001-U-lsg.eps}
\end{examaufgabe}

\begin{examaufgabe}{Fließbild}{20}{KT}{PF}{RI}
\end{examaufgabe}

\begin{examaufgabe}{Funktionsanalyse}{20}{KT}{PF}{KTFUNKTION}
\anhang{\hlogpdia}{r134a}
\end{examaufgabe}

\begin{examaufgabe}{Wärmeübertrager}{20}{KT}{PF}{WU}
\end{examaufgabe}

\begin{examaufgabe}{Verdichter}{20}{KT}{PF}{VERDICHTER}
\end{examaufgabe}

\begin{examaufgabe}{\hxdia}{20}{KL}{PF}{HX}
\anhang{\hxdia}{hx-diagram-bw.ps}
\end{examaufgabe}

\begin{examaufgabe}{Feuchte Luft}{20}{KL}{PF}{LUFT}
\end{examaufgabe}

\begin{examaufgabe}{MOP-Ventil}{10}{KT}{WA}{MOP}
\end{examaufgabe}

\begin{examaufgabe}{COP-Wert}{10}{KT}{WA}{COP}
\end{examaufgabe}

\begin{examaufgabe}{Kühlturm}{10}{KT}{WA}{KUEHLTURM}
\end{examaufgabe}

\begin{examaufgabe}{Rohrleitungsberechnung}{10}{KT}{WA}{ROHR}
\end{examaufgabe}

\begin{examaufgabe}{Schaltungsanalyse}{20}{ET}{PF}{SCHALTUNG}
\end{examaufgabe}

\begin{examaufgabe}{Elektro-Schaltplan}{20}{ET}{PF}{SCHALTPLAN}
\end{examaufgabe}

\begin{examaufgabe}{Elektromotor}{20}{ET}{PF}{MOTOR}
\end{examaufgabe}

\begin{examaufgabe}{Drehstrom}{20}{ET}{PF}{DREHSTROM}
\end{examaufgabe}

\begin{examaufgabe}{Schutz vor Strom}{20}{ET}{PF}{ETSCHUTZ}
\end{examaufgabe}

\begin{examaufgabe}{Kältekreislauf}{5}{KT}{PF}{KREISLAUF}
\end{examaufgabe}

\begin{examaufgabe}{Fließbild}{5}{KT}{PF}{RI}
\end{examaufgabe}

\begin{examaufgabe}{Druckschalter}{5}{KT}{PF}{PRESSOSTAT}
\end{examaufgabe}

\begin{examaufgabe}{Kältemittel}{5}{KT}{PF}{REFRIG}
\end{examaufgabe}

\begin{examaufgabe}{Gestreckte Länge}{5}{KT}{PF}{GESLAENGE}
\end{examaufgabe}

\begin{examaufgabe}{Isometrische Darstellung}{5}{KT}{PF}{ISOMETRIE}
\end{examaufgabe}

\begin{examaufgabe}{Wärmelehre}{5}{KT}{PF}{WAERME}
\end{examaufgabe}

\begin{examaufgabe}{\hxdia}{5}{KT}{PF}{HX}
\end{examaufgabe}

\begin{examaufgabe}{Grundlagen Elektrotechnik}{5}{ET}{PF}{ETBASIS}
\end{examaufgabe}

\begin{examaufgabe}{Elektrotechnik-Symbole}{5}{ET}{PF}{ETSYMBOLE}
\end{examaufgabe}

\begin{examaufgabe}{Stromkosten}{5}{ET}{PF}{ETKOSTEN}
\end{examaufgabe}

\end{document}
