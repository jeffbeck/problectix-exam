\documentclass[bsz-mkk-exam,ka]{teacher}
\Titelo{KA}
\Titelu{Nr. 1}
\Datum{12.12.2016}
\Fach{Lernbereich Theorie}
\Klasse{M1KK3T}
\UserToken{BZ}
\gruppea
\begin{document}

%\BTone{final}
\BTtwo{draft}

\begin{examaufgabe}[T]{\hlogpdia}{20}{KT}{PF}{HLOGP}
\end{examaufgabe}

\begin{examaufgabe}[T]{\hxdia}{20}{KL}{PF}{HX}
\end{examaufgabe}

\begin{examaufgabe}[T]{Fließbild}{20}{KT}{PF}{RI}
\end{examaufgabe}

\begin{examaufgabe}[T]{Funktionsanalyse}{20}{KT}{PF}{KTFUNKTION}
\end{examaufgabe}

\begin{examaufgabe}[T]{Wärmeübertrager}{20}{KT}{PF}{WU}
\end{examaufgabe}

\begin{examaufgabe}[T]{Verdichter}{20}{KT}{PF}{VERDICHTER}
\end{examaufgabe}

\begin{examaufgabe}[T]{Feuchte Luft}{20}{KL}{PF}{LUFT}
\end{examaufgabe}

\begin{examaufgabe}[T]{MOP-Ventil}{10}{ET}{WA}{MOP}
\end{examaufgabe}

\begin{examaufgabe}[T]{COP-Wert}{10}{KT}{WA}{COP}
\end{examaufgabe}

\begin{examaufgabe}[T]{Kühlturm}{10}{KT}{WA}{KUEHLTURM}
\end{examaufgabe}

\begin{examaufgabe}[T]{Rohrleitungsberechnung}{10}{KT}{WA}{ROHR}
\end{examaufgabe}

\begin{examaufgabe}[T]{Schaltungsanalyse}{20}{ET}{PF}{SCHALTUNG}
\end{examaufgabe}

\begin{examaufgabe}[T]{Elektro-Schaltplan}{20}{ET}{PF}{SCHALTPLAN}
\end{examaufgabe}

\begin{examaufgabe}[T]{Elektromotor}{20}{ET}{PF}{MOTOR}
\end{examaufgabe}

\begin{examaufgabe}[T]{Drehstrom}{20}{ET}{PF}{DREHSTROM}
\end{examaufgabe}

\begin{examaufgabe}[T]{Schutz vor Strom}{20}{ET}{PF}{ETSCHUTZ}
\end{examaufgabe}

\begin{examaufgabe}[T]{Kältekreislauf}{5}{KT}{PF}{KREISLAUF}
\end{examaufgabe}

\begin{examaufgabe}[T]{Druckschalter}{5}{KT}{PF}{PRESSOSTAT}
\end{examaufgabe}

\begin{examaufgabe}[T]{Kältemittel}{5}{KT}{PF}{REFRIG}
\end{examaufgabe}

\begin{examaufgabe}[T]{Gestreckte Länge}{5}{KT}{PF}{GESLAENGE}
\end{examaufgabe}

\begin{examaufgabe}[T]{Isometrische Darstellung}{5}{KT}{PF}{ISOMETRIE}
\end{examaufgabe}

\begin{examaufgabe}[T]{Wärmelehre}{5}{KT}{PF}{WAERME}
\end{examaufgabe}

\begin{examaufgabe}[T]{Grundlagen Elektrotechnik}{5}{ET}{PF}{ETBASIS}
\end{examaufgabe}

\begin{examaufgabe}[T]{Elektrotechnik-Symbole}{5}{ET}{PF}{ETSYMBOLE}
\end{examaufgabe}

\begin{examaufgabe}[T]{Stromkosten}{5}{ET}{PF}{ETKOSTEN}
\end{examaufgabe}

\end{document}
