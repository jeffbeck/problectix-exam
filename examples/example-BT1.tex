\documentclass[bsz-mkk-exam,ka]{teacher}
\Titelo{KA}
\Titelu{Nr. 1}
\Datum{12.12.2016}
\Fach{Lernbereich Theorie}
\Klasse{M1KK3T}
\UserToken{BZ}
\gruppea
\begin{document}

%\BTone{final}
\BTtwo{draft}

\begin{aufgabe}{Abkühlen von Luft}
\examinfo{12}{ET}{WA}{}
   \begin{textonly}
     Die Luft in einem Lagerraum hat eine Temperatur von
     \unit{25}{\celsius} und eine relative Luftfeuchte $\varphi$ von
     \unit{50}{\%}.
   \end{textonly}
   \begin{teilaufgabe}{ap}{1}{2}
     Auf welche minimale Temperatur kann diese Luft ohne Entfeuchtung
     abgekühlt werden?
   \end{teilaufgabe}
   \begin{loesung}
     \punkte{Luftzustand, \unit{14}{\celsius}}{2}{}
   \end{loesung}
   \begin{teilaufgabe}{ap}{1}{1}
      Wie wird diese Temperatur genannt?
   \end{teilaufgabe}
   \begin{loesung}
     \punkte{Taupunkttemperatur}{1}{}
   \end{loesung}
   \begin{teilaufgabe}{ap}{1}{1}
     Welche spezifische Enthalpie muss der Luft bei dieser Akühlung
     entzogen werden?
   \end{teilaufgabe}
   \begin{loesung}
     \punkte{$\Delta h = 51 - 39 = \unit{12}{\kilo\joule\per\kilo\gram}$}{1}{}
   \end{loesung}
\end{aufgabe}


\end{document}
