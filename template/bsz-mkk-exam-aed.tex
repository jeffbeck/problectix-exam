\newcounter{aufgabenanzahl}
% calculate numer of problems issud
\setcounter{aufgabenanzahl}{\theaufgabennummer-1}

% Ausgabe einer counter Tabellenzeile
\newcommand{\countertabline}[4]{ #4 & 
                                 #2 & 
                                 \ifthenelse{\equal{#2}{#3}}%
                                            {\textcolor{gruen}{#3}}%
                                            {\textcolor{rot}{#3}} & 
                                 #1 \\ \hline}


% printout in draft mode
\ifthenelse{\value{ModeFinal}=1}{}{% doing nothing on final
\newpage

\textbf{Resulting counters for }\ExamWhich:

\bigskip

   \renewcommand{\arraystretch}{1.2}
     \begin{tabular}{|l|r|r|l|} \hline
       Beschreibung (Zeit in Minuten) & Sollwert & Istwert & Zähler-Name\\ \hline\hline
       \countertabline{TimeTotal}{\theTimeTotalOK}{\theTimeTotal}{Gesamtzeit \ExamWhich}
       \countertabline{TimeTotalKT}{\theTimeTotalKTOK}{\theTimeTotalKT}{Gesamtzeit KT (Kältetechnik) \ExamWhich}
       \countertabline{TimeTotalET}{\theTimeTotalETOK}{\theTimeTotalET}{Gesamtzeit ET (Elektrotechnik) \ExamWhich}
       \countertabline{TimeTotalPF}{\theTimeTotalPFOK}{\theTimeTotalPF}{Gesamtzeit PF (Pflichtbereich) in \ExamWhich}
       \countertabline{TimeTotalWA}{\theTimeTotalWAOK}{\theTimeTotalWA}{Gesamtzeit WA (Wahlbereich) \ExamWhich}
       \countertabline{TimeWAError}{\theTimeWAErrorOK}{\theTimeWAError}{WA-Aufgabe nicht \theTimeWAOK{} Minuten in \ExamWhich}
       \countertabline{NumberTotalWAOK}{\theNumberTotalWAOK}{\theNumberTotalWA}{Anzahl Wahlaufgaben in \ExamWhich}
       \countertabline{ExamInfo}{\theaufgabenanzahl}{\theExamInfo}{\theaufgabenanzahl{} Aufgaben mit \theExamInfo{} Zusatzinfos}
     \end{tabular}
   \renewcommand{\arraystretch}{1.0}

\bigskip

\textbf{TopicTime counters for }\ExamWhich:

\bigskip

   \renewcommand{\arraystretch}{1.2}
     \begin{tabular}{|l|r|r|l|} \hline
       Beschreibung & MinMinutes & Istwert & Zähler-Name\\ \hline\hline
       \textbf{BT1/BT2 KT} &  & & \\ \hline\hline
       \countertabline{TopicTimeHX}{\theTopicTimeHXMin}{\theTopicTimeHX}{\hxdia}
       \countertabline{TopicTimeLOGPH}{\theTopicTimeLOGPHMin}{\theTopicTimeLOGPH}{\hlogpdia}
       \countertabline{TopicTimeRI}{\theTopicTimeRIMin}{\theTopicTimeRI}{RI-Fließbild}
       \countertabline{TopicTimeMOP}{\theTopicTimeMOPMin}{\theTopicTimeMOP}{MOP}
       \countertabline{TopicTimeKTFUNKTION}{\theTopicTimeKTFUNKTIONMin}{\theTopicTimeKTFUNKTION}{Funktionsanalyse Kältetechnik}
       \countertabline{TopicTimeWU}{\theTopicTimeWUMin}{\theTopicTimeWU}{Wärmeübertrager}
       \countertabline{TopicTimeVERDICHTER}{\theTopicTimeVERDICHTERMin}{\theTopicTimeVERDICHTER}{Verdichter}
       \countertabline{TopicTimeFEUCHTELUFT}{\theTopicTimeFEUCHTELUFTMin}{\theTopicTimeFEUCHTELUFT}{Feuchte Luft}
       \countertabline{TopicTimeCOP}{\theTopicTimeCOPMin}{\theTopicTimeCOP}{COP-Wert}
       \countertabline{TopicTimeKUEHLTURM}{\theTopicTimeKUEHLTURMMin}{\theTopicTimeKUEHLTURM}{Kühlturm}
       \countertabline{TopicTimeROHRLEITUNG}{\theTopicTimeROHRLEITUNGMin}{\theTopicTimeROHRLEITUNG}{Rohrleitungen}\hline
       \textbf{BT1/BT2 KT} &  & & \\ \hline\hline
       \countertabline{TopicTimeSCHALTUNG}{\theTopicTimeSCHALTUNGMin}{\theTopicTimeSCHALTUNG}{Schaltungsanalyse ET}
       \countertabline{TopicTimeSCHALTPLAN}{\theTopicTimeSCHALTPLANMin}{\theTopicTimeSCHALTPLAN}{Schaltplan ET}
       \countertabline{TopicTimeMOTOR}{\theTopicTimeMOTORMin}{\theTopicTimeMOTOR}{Elektromotor ET}
       \countertabline{TopicTimeDREHSTROM}{\theTopicTimeDREHSTROMMin}{\theTopicTimeDREHSTROM}{Drehstrom ET}
       \countertabline{TopicTimeSTROMSCHUTZ}{\theTopicTimeSTROMSCHUTZMin}{\theTopicTimeSTROMSCHUTZ}{Schutz vor elektr. Strom ET}\hline
       \textbf{ZP KT} &  & & \\ \hline\hline
       \countertabline{TopicTimeKREISLAUF}{\theTopicTimeKREISLAUFMin}{\theTopicTimeKREISLAUF}{Kältekreislauf ZP}
       \countertabline{TopicTimeDRUCKSCHALTER}{\theTopicTimeDRUCKSCHALTERMin}{\theTopicTimeDRUCKSCHALTER}{Druckschalter ZP}
       \countertabline{TopicTimeREFRIGERANT}{\theTopicTimeREFRIGERANTMin}{\theTopicTimeREFRIGERANT}{Kältemittel ZP}
       \countertabline{TopicTimeGESLAENGE}{\theTopicTimeGESLAENGEMin}{\theTopicTimeGESLAENGE}{Gestreckte Länge ZP}
       \countertabline{TopicTimeISOMETRIE}{\theTopicTimeISOMETRIEMin}{\theTopicTimeISOMETRIE}{Isometrische Darstellung ZP}
       \countertabline{TopicTimeWAERME}{\theTopicTimeWAERMEMin}{\theTopicTimeWAERME}{Wärmeübertragung ZP}\hline
       \textbf{ZP ET} &  & & \\ \hline\hline
       \countertabline{TopicTimeETBASIS}{\theTopicTimeETBASISMin}{\theTopicTimeETBASIS}{Elektrotechnik Grundlagen ZP}
       \countertabline{TopicTimeETSYMBOLE}{\theTopicTimeETSYMBOLEMin}{\theTopicTimeETSYMBOLE}{Elektrotechnik Symbole ZP}
       \countertabline{TopicTimeETKOSTEN}{\theTopicTimeETKOSTENMin}{\theTopicTimeETKOSTEN}{Elektrotechnik Kosten ZP}
     \end{tabular}
   \renewcommand{\arraystretch}{1.0}

}%
