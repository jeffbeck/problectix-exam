\newcounter{aufgabenanzahl}
% calculate numer of problems issud
\setcounter{aufgabenanzahl}{\theaufgabennummer-1}

% Ausgabe einer counter Tabellenzeile
\newcommand{\countertabline}[4]{ #4 & 
                                 #2 & 
                                 \ifthenelse{\equal{#2}{#3}}%
                                            {\textcolor{gruen}{#3}}%
                                            {\textcolor{rot}{#3}} & 
                                 #1 \\ \hline}

% Ausgabe einer counter Tabellenzeile
\newcommand{\topictabline}[4]{ #5 & 
                               #2 & 
                               #3 & 
                               #4 &
%                               \ifthenelse{\equal{#2}{#3}}%
%                                          {\textcolor{gruen}{#4}}%
%                                          {\textcolor{rot}{#4}} & 
                               #1 \\ \hline}


% printout in draft mode
\ifthenelse{\value{ModeFinal}=1}{}{% doing nothing on final
%\newpage

\textbf{Resulting counters for }\ExamWhich:

\bigskip

   \renewcommand{\arraystretch}{1.25}
     \begin{tabular}{|l|r|r|l|} \hline
       Beschreibung (Zeit in Minuten) & Sollwert & Istwert & Zähler-Name\\ \hline\hline
       \countertabline{TimeTotal}{\theTimeTotalOK}{\theTimeTotal}{Gesamtzeit \ExamWhich}
       \countertabline{TimeTotalKT}{\theTimeTotalKTOK}{\theTimeTotalKT}{Gesamtzeit KT (Kältetechnik) \ExamWhich}
       \countertabline{TimeTotalET}{\theTimeTotalETOK}{\theTimeTotalET}{Gesamtzeit ET (Elektrotechnik) \ExamWhich}
       \countertabline{TimeTotalPF}{\theTimeTotalPFOK}{\theTimeTotalPF}{Gesamtzeit PF (Pflichtbereich) in \ExamWhich}
       \countertabline{TimeTotalWA}{\theTimeTotalWAOK}{\theTimeTotalWA}{Gesamtzeit WA (Wahlbereich) \ExamWhich}
       \countertabline{TimeWAError}{\theTimeWAErrorOK}{\theTimeWAError}{WA-Aufgabe nicht \theTimeWAOK{} Minuten in \ExamWhich}
       \countertabline{NumberTotalWAOK}{\theNumberTotalWAOK}{\theNumberTotalWA}{Anzahl Wahlaufgaben in \ExamWhich}
       \countertabline{ExamInfo}{\theaufgabenanzahl}{\theExamInfo}{\theaufgabenanzahl{} Aufgaben mit \theExamInfo{} Zusatzinfos}
     \end{tabular}
   \renewcommand{\arraystretch}{1.0}

\textbf{Resulting counters for }\ExamWhich:

\bigskip

   \renewcommand{\arraystretch}{1.25}
     \begin{tabular}{|l|r|r|l|} \hline
       Beschreibung & Min & Max & Istwert & Zähler-Name\\ \hline\hline
       \topictabline{TopicHX}{\theTopicHXMin}{\theTopicHXMax}{\theTopicHX}{Aufgaben hx-Diagramm}

     \end{tabular}
   \renewcommand{\arraystretch}{1.0}

}%
