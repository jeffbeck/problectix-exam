% newpage and the rest if in draft mode only
%\newpage
\newcounter{aufgabenanzahl}
% calculate numer of problems issud
\setcounter{aufgabenanzahl}{\theaufgabennummer-1}

% Ausgabe einer Tabellenzeile
\newcommand{\countertabline}[4]{ #4 & 
                                 #2 & 
                                 \ifthenelse{\equal{#2}{#3}}%
                                            {\textcolor{gruen}{#3}}%
                                            {\textcolor{rot}{#3}} & 
                                 #1 \\ \hline}



% printout
Resulting counters for \ExamWhich:

\bigskip

   \renewcommand{\arraystretch}{1.25}
     \begin{tabular}{|l|r|r|l|} \hline
       Beschreibung (Zeit in Minuten) & Sollwert & Istwert & Zähler-Name\\ \hline\hline
       \countertabline{TimeTotal}{\theTimeTotalOK}{\theTimeTotal}{Gesamtzeit in Minuten \ExamWhich}
       \countertabline{TimeTotalKT}{\theTimeTotalKTOK}{\theTimeTotalKT}{Gesamtzeit KT in Minuten \ExamWhich}
       \countertabline{TimeTotalET}{\theTimeTotalETOK}{\theTimeTotalET}{Gesamtzeit ET in Minuten \ExamWhich}
       \countertabline{TimeTotalPF}{\theTimeTotalPFOK}{\theTimeTotalPF}{Gesamtzeit PF in \ExamWhich}
       \countertabline{TimeTotalWA}{\theTimeTotalWAOK}{\theTimeTotalWA}{Gesamtzeit WA in \ExamWhich}
       \countertabline{TimeWAError}{\theTimeWAErrorOK}{\theTimeWAError}{WA nicht \theTimeWAOK{} Minuten in \ExamWhich}
       \countertabline{NumberTotalWAOK}{\theNumberTotalWAOK}{\theNumberTotalWA}{Anzahl Wahlaufgaben in \ExamWhich}
       \countertabline{ExamInfo}{\theaufgabenanzahl}{\theExamInfo}{\theaufgabenanzahl{} Aufgaben mit \theExamInfo{} Zusatzinfos}
     \end{tabular}
   \renewcommand{\arraystretch}{1.0}
