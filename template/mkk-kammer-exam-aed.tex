\newcounter{aufgabenanzahl}
% calculate numer of problems issued
\setcounter{aufgabenanzahl}{\theaufgabennummer-1}

% Ausgabe einer counter Tabellenzeile
\newcommand{\counterequal}[4]{ #4 & 
                               #2 & 
                               \ifthenelse{\equal{#2}{#3}}%
                                          {\textcolor{gruen}{#3}}%
                                          {\textcolor{rot}{#3}} & 
                               #1 \\ \hline}
\newcommand{\countermin}[4]{ #4 & 
                             #2 & 
                             \ifthenelse{#3<#2}%
                                        {\textcolor{rot}{#3}}%
                                        {\textcolor{gruen}{#3}} & 
                             #1 \\ \hline}


% printout in draft mode
\ifthenelse{\value{ModeFinal}=1}{}{% doing nothing on final
\newpage

\textbf{Resulting counters for }\ExamWhich{} (Beim Korrekturlesen ignorieren!):

\bigskip

   \renewcommand{\arraystretch}{1.25}
     \begin{tabular}{|l|r|r|l|} \hline
       Beschreibung (Zeit in Minuten) & Sollwert & Istwert & Zähler-Name\\ \hline\hline
       \counterequal{TimeTotal}{\theTimeTotalOK}{\theTimeTotal}{Gesamtzeit \ExamWhich}\hline
       \counterequal{TimeTotalKT}{\theTimeTotalKTOK}{\theTimeTotalKT}{Gesamtzeit KT (Kältetechnik) \ExamWhich}
       \counterequal{TimeTotalKL}{\theTimeTotalKLOK}{\theTimeTotalKL}{Gesamtzeit KL (Klimatechnik) \ExamWhich}
       \counterequal{TimeTotalET}{\theTimeTotalETOK}{\theTimeTotalET}{Gesamtzeit ET (Elektrotechnik) \ExamWhich}\hline
       \counterequal{TimeTotalPF}{\theTimeTotalPFOK}{\theTimeTotalPF}{Gesamtzeit PF (Pflichtbereich) in \ExamWhich}
       \counterequal{TimeTotalWA}{\theTimeTotalWAOK}{\theTimeTotalWA}{Gesamtzeit WA (Wahlbereich) \ExamWhich}\hline
       \counterequal{TimeWAError}{\theTimeWAErrorOK}{\theTimeWAError}{WA-Aufgabe nicht \theTimeWAOK{} Minuten in \ExamWhich}
       \counterequal{NumberTotalWAOK}{\theNumberTotalWAOK}{\theNumberTotalWA}{Anzahl Wahlaufgaben in \ExamWhich}
       \counterequal{ExamInfo}{\theaufgabenanzahl}{\theExamInfo}{\theaufgabenanzahl{} Aufgaben mit \theExamInfo{} Zusatzinfos}
       \counterequal{NumberTopicsCounted}{\theaufgabenanzahl}{\theNumberTopicsCounted}{\theaufgabenanzahl{} Aufgaben mit Topic-Zähler  \theNumberTopicsCounted{}-mal erhöht}
     \end{tabular}
   \renewcommand{\arraystretch}{1.0}

\newpage

\textbf{TopicTime counters for }\ExamWhich{} (Beim Korrekturlesen ignorieren!):

\bigskip

   \renewcommand{\arraystretch}{1.25}
     \begin{tabular}{|l|r|r|l|} \hline
       Beschreibung & MinMinutes & Istwert & Zähler-Name\\ \hline\hline
       \textbf{BT1/BT2 KT} &  & & \\ \hline\hline
       \countermin{TopicTimeHLOGP}{\theTopicTimeHLOGPMin}{\theTopicTimeHLOGP}{\hlogpdia}
       \countermin{TopicTimeHX}{\theTopicTimeHXMin}{\theTopicTimeHX}{\hxdia}
       \countermin{TopicTimeRI}{\theTopicTimeRIMin}{\theTopicTimeRI}{RI-Fließbild}
       \countermin{TopicTimeKTFUNKTION}{\theTopicTimeKTFUNKTIONMin}{\theTopicTimeKTFUNKTION}{Funktionsanalyse Kältetechnik}
       \countermin{TopicTimeWU}{\theTopicTimeWUMin}{\theTopicTimeWU}{Wärmeübertrager}
       \countermin{TopicTimeVERDICHTER}{\theTopicTimeVERDICHTERMin}{\theTopicTimeVERDICHTER}{Verdichter}
       \countermin{TopicTimeLUFT}{\theTopicTimeLUFTMin}{\theTopicTimeLUFT}{Feuchte Luft}
       \countermin{TopicTimeMOP}{\theTopicTimeMOPMin}{\theTopicTimeMOP}{MOP}
       \countermin{TopicTimeABTAUUNG}{\theTopicTimeABTAUUNGMin}{\theTopicTimeABTAUUNG}{ABTAUUNG}
       \countermin{TopicTimeCOP}{\theTopicTimeCOPMin}{\theTopicTimeCOP}{COP-Wert}
       \countermin{TopicTimeKUEHLTURM}{\theTopicTimeKUEHLTURMMin}{\theTopicTimeKUEHLTURM}{Kühlturm}
       \countermin{TopicTimeROHR}{\theTopicTimeROHRMin}{\theTopicTimeROHR}{Rohrleitungen}
       \countermin{TopicTimeREGLER}{\theTopicTimeREGLERMin}{\theTopicTimeREGLER}{Regler}\hline
       \textbf{BT1/BT2 ET} &  & & \\ \hline\hline
       \countermin{TopicTimeSCHALTUNG}{\theTopicTimeSCHALTUNGMin}{\theTopicTimeSCHALTUNG}{Schaltungsanalyse ET}
       \countermin{TopicTimeSCHALTPLAN}{\theTopicTimeSCHALTPLANMin}{\theTopicTimeSCHALTPLAN}{Schaltplan ET}
       \countermin{TopicTimeMOTOR}{\theTopicTimeMOTORMin}{\theTopicTimeMOTOR}{Elektromotor ET}
       \countermin{TopicTimeDREHSTROM}{\theTopicTimeDREHSTROMMin}{\theTopicTimeDREHSTROM}{Drehstrom ET}
       \countermin{TopicTimeETSCHUTZ}{\theTopicTimeETSCHUTZMin}{\theTopicTimeETSCHUTZ}{Schutz vor elektr. Strom ET}
       \countermin{TopicTimeETBAUTEILE}{\theTopicTimeETBAUTEILEMin}{\theTopicTimeETBAUTEILE}{Elektrische Bauteile}\hline
       \textbf{ZP KT} &  & & \\ \hline\hline
       \countermin{TopicTimeKREISLAUF}{\theTopicTimeKREISLAUFMin}{\theTopicTimeKREISLAUF}{Kältekreislauf ZP}
       \countermin{TopicTimePRESSOSTAT}{\theTopicTimePRESSOSTATMin}{\theTopicTimePRESSOSTAT}{Druckschalter ZP}
       \countermin{TopicTimeREFRIG}{\theTopicTimeREFRIGMin}{\theTopicTimeREFRIG}{Kältemittel ZP}
       \countermin{TopicTimeGESLAENGE}{\theTopicTimeGESLAENGEMin}{\theTopicTimeGESLAENGE}{Gestreckte Länge ZP}
       \countermin{TopicTimeISOMETRIE}{\theTopicTimeISOMETRIEMin}{\theTopicTimeISOMETRIE}{Isometrische Darstellung ZP}
       \countermin{TopicTimeWAERME}{\theTopicTimeWAERMEMin}{\theTopicTimeWAERME}{Wärmeübertragung ZP}\hline
       \textbf{ZP ET} &  & & \\ \hline\hline
       \countermin{TopicTimeETBASIS}{\theTopicTimeETBASISMin}{\theTopicTimeETBASIS}{Elektrotechnik Grundlagen ZP}
       \countermin{TopicTimeETSYMBOLE}{\theTopicTimeETSYMBOLEMin}{\theTopicTimeETSYMBOLE}{Elektrotechnik Symbole ZP}
       \countermin{TopicTimeETKOSTEN}{\theTopicTimeETKOSTENMin}{\theTopicTimeETKOSTEN}{Elektrotechnik Kosten ZP}
     \end{tabular}
   \renewcommand{\arraystretch}{1.0}

}%

\addtocontents{toc}{\vspace{3mm}\textbf{Anhang}\par}
